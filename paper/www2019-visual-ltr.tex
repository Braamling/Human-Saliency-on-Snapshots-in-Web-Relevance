\pdfoutput=1

\documentclass[sigconf,screen=true]{acmart} 
% \documentclass[sigconf,screen=true]{acmart} 

%%%%%

% Copyright
%\setcopyright{none}
%\setcopyright{acmcopyright}
%\setcopyright{acmlicensed}
\setcopyright{rightsretained}
%\setcopyright{usgov}
%\setcopyright{usgovmixed}
%\setcopyright{cagov}
%\setcopyright{cagovmixed}

% DOI
% \acmDOI{10.475/123_4}
% ISBN
% \acmISBN{123-4567-24-567/08/06}

\acmConference[WWW 2019]{The Web Conference 2019:  The 28th International World Wide Web Conference}{May 13-17, 2019}{San Francisco}
\acmYear{2019}
\copyrightyear{2019}

\settopmatter{printfolios=false,printacmref=true}
\fancyhead{}

% Space hacks
%\clubpenalty=10000
%\widowpenalty=10000
\usepackage[subtle, wordspacing=normal, tracking=normal, bibnotes, charwidths=normal, leading, indent, lists, paragraphs=normal, mathspacing=normal, bibliography=normal]{savetrees}
%\usepackage[subtle]{savetrees}
%\looseness=-1

\usepackage{hyperref}
\usepackage{booktabs} % For formal tables
\usepackage{subfig}
\usepackage[skip=0pt]{caption}
\usepackage{graphicx}
\usepackage{paralist}
\usepackage{enumitem}
\usepackage{acronym}
\usepackage{amsmath}
%\newcommand{\todo}[1]{\textcolor{red}{TODO: #1}}
%\newcommand{\im}[1]{\textcolor{blue}{IM: #1}}
%\newcommand{\mdr}[1]{\textcolor{cyan}{#1}}
\acrodef{LTR}{learning to rank}
\acrodef{ViTOR}{\textit{Visual learning TO Rank}}
\acrodef{SERP}{search engine results page}

\newcommand{\datasetname}{\ac{ViTOR}}
\newcommand{\modelname}{\ac{ViTOR}}
\newcommand{\modelnamef}{\acf{ViTOR}}
\newcommand{\OK}{@{\mbox{}\hspace*{.25cm}}}

\title{ViTOR: Learning to Rank Webpages Based on Visual Features}
% \subtitle{An Improved Method and a Dataset}

\author{Bram van den Akker}
\orcid{}
\affiliation{%
  \institution{University of Amsterdam}
  \city{Amsterdam} 
  \country{The Netherlands}
}
\email{contact@bramvandenakker.nl}

\author{Ilya Markov}
\orcid{}
\affiliation{%
  \institution{University of Amsterdam}
  \city{Amsterdam} 
  \country{The Netherlands}  
}
\email{i.markov@uva.nl}

\author{Maarten de Rijke}
\orcid{0000-0002-1086-0202}
\affiliation{%
   \institution{University of Amsterdam}
   \city{Amsterdam} 
   \country{The Netherlands}
}
\email{derijke@uva.nl}

% The default list of authors is too long for headers.
\renewcommand{\shortauthors}{B. van den Akker et al.}

\begin{document}

%Webpage \ac{LTR} based on visual features is a method of ranking (textual) webpages for a query using their visual appearance as (additional) features.  
%
%In this work we improve webpage \ac{LTR} based on visual features, a method of ranking webpages for a query using their visual appearance as (additional) features. 
%
%\mdr{In \ac{LTR} webpages using visual features one uses the visual appearance of webpages as (additional) features to rank (textual) webpages in response to a query.}
\begin{abstract}
The visual appearance of a webpage carries valuable information about page's quality and can be used to improve the performance of \ac{LTR}.
%Features learned from the visual appearance of a webpage can help to improve \ac{LTR} performance. 
We introduce the \modelname~model that integrates state-of-the-art visual features extraction methods:
%We implement the proposed \modelname~model by adopting two powerful visual feature extraction methods:
\begin{inparaenum}[(i)]
\item transfer learning from a pre-trained image classification model, and
\item synthetic saliency heatmaps generated from webpage snapshots.
\end{inparaenum}
Since there is currently no public dataset available for the task of \ac{LTR} with visual features, we also introduce and release the \datasetname~dataset, containing visually rich and diverse webpages.
The \datasetname~dataset consists of visual snapshots, non-visual features and relevance judgments for ClueWeb12 webpages and TREC Web Track queries.
We experiment with the proposed \modelname~model on the newly introduced \datasetname~dataset
and show that our model significantly improves the performance of \ac{LTR} with visual features.
%The visual snapshots are rendered with both styling and images using the Wayback Machine and ClueWeb12 online rendering service.
%Using \datasetname, we confirm that visual features can improve \ac{LTR} performance
%and show the effectiveness of the newly introduced architecture and visual feature vectors.
% Additionally, we show that visual features can be used efficiently with a merely neglectable increase in computational cost.
\end{abstract}

%
% The code below should be generated by the tool at
% http://dl.acm.org/ccs.cfm
% Please copy and paste the code instead of the example below. 
%
% \begin{CCSXML}
% <ccs2012>
%  <concept>
%   <concept_id>10010520.10010553.10010562</concept_id>
%   <concept_desc>Computer systems organization~Embedded systems</concept_desc>
%   <concept_significance>500</concept_significance>
%  </concept>
%  <concept>
%   <concept_id>10010520.10010575.10010755</concept_id>
%   <concept_desc>Computer systems organization~Redundancy</concept_desc>
%   <concept_significance>300</concept_significance>
%  </concept>
%  <concept>
%   <concept_id>10010520.10010553.10010554</concept_id>
%   <concept_desc>Computer systems organization~Robotics</concept_desc>
%   <concept_significance>100</concept_significance>
%  </concept>
%  <concept>
%   <concept_id>10003033.10003083.10003095</concept_id>
%   <concept_desc>Networks~Network reliability</concept_desc>
%   <concept_significance>100</concept_significance>
%  </concept>
% </ccs2012>  
% \end{CCSXML}

% \ccsdesc[500]{Computer systems organization~Embedded systems}
% \ccsdesc[300]{Computer systems organization~Redundancy}
% \ccsdesc{Computer systems organization~Robotics}
% \ccsdesc[100]{Networks~Network reliability}


\keywords{Learning to rank, Visual features}


\maketitle

% !TEX root = cikm2018-visual-ltr.tex

\section{Introduction}
Over the years, various research demonstrated the impact of web design on how users find content on the web~\cite{nielsen1999designing,nielsen2006f,pernice2017f,wang2014eye}.
Although web design has become a 34 billion dollar industry in the US alone according to \citet{ibisdesign}, modern search engines are still focused on using content features (e.g., BM25, PageRank, etc.) to determine web relevance \todo{refs?}. 

Recently \citet{fan2017learning} introduced ViP, a \ac{LTR} model that uses a combination of both visual and content features.
Visual features are calculated from snapshots, which are in turn created by rendering web pages.
The authors demonstrate that adding these visual features significantly improve the \ac{LTR} performance.
The indication that the visual representation of a web page can have a significant impact on web search leverages a new field in \ac{LTR} research.

However, there are a few limitations in \cite{fan2017learning}.
First, the rendered web pages come from the the GOV2 collection~\todo{ref}.\footnote{\url{http://ir.dcs.gla.ac.uk/test_collections/gov2-summary.htm}}
This collection is limited in the sense that:
\begin{inparaenum}[(i)]
\item it solely contains web pages within the .gov domain crawled in 2002, i.e., somewhat outdated pages with a relatively narrow scope,
\item original images are not included in the pages contained in the dataset, and
\item the styling information is not available together with web pages.
\end{inparaenum}
We believe that in order to advance research on \ac{LTR} with visual features it is required to have a dataset with more diverse and up-to-date documents and richer visual information (e.g., styling, images, etc).

Second limitation is \todo{a suboptimal / not state-of-the-art visual features extraction. Describe this in detail similarly to the above.}

In this work, we attempt to address the above limitations.
First, we propose \datasetname, a dataset that \todo{describe key features of the proposed dataset by unfolding the next sentence into a few sentences}
\datasetname is a combination of queries and content features from TREC Web 2013 \& 2014 together with screenshots from ClueWeb12 documents. 
We believe that this dataset allows the \ac{LTR} research community to investigate various methods of using visual features for search and ranking.
In particular, using \datasetname, we show that the results of the ViP model from \citet{fan2017learning} can be reproduced on a more diverse dataset.

Second, \todo{something about the state-of-the-art in visual features}
In particular, we use the following two state-of-the-art methods to extract visual features from snapshots:
\begin{inparaenum}[(i)]
\item transfer learning from a pre-trained image recognition model~\cite{donahue2014decaf,simonyan2014very}, and
\item generation of synthetic saliency heatmaps from the web page snapshots~\cite{shen2014webpage,shan2017two}.
\end{inparaenum}
We show that both methods are able to improve retrieval performance significantly.

The main contribution of this work are:
\begin{enumerate}  
\item We propose and publish \datasetname, an out-of-the-box dataset for \ac{LTR} with visual features.
\item We reproduce the ViP model from \cite{fan2017learning} on the newly proposed dataset.
\item We improve ViP by \todo{XXX and YYY}.
\end{enumerate}

\todo{Revise this if we have space or drop it.} The rest of the paper is organized as follows. Section~\ref{sec:dataset} describes the collection process that was used to construct \datasetname. In Section~\ref{sec:experiments} we reproduce the work of \citet{fan2017learning} and demonstrate various feature extraction methods on \datasetname and set a baseline for future visual \ac{LTR} research.  
% !TEX root = cikm2018-visual-ltr.tex

\section{Related work}\label{sec:relatedwork}
In this section, we discuss research on
\begin{inparaenum}[(i)]
\item the relation between visual appearance of a web page and how it is perceived by users, and
\item the usage of visual information for web page ranking.
\end{inparaenum} 

%The work in the paper is related to various research on user experience design. In this section we will discuss research on:
%\begin{inparaenum}[(i)]
%\item measuring usability, 
%\item predicting saliency on web pages, and 
%\item visual features in \ac{LTR}.
%\end{inparaenum} 

Using eye-tracking, \citet{nielsen2006f} and \citet{pernice2017f} demonstrate that web page design and content placement influences the ability for users to find the information they are looking for. 
Both studies show that by organizing the content in certain shapes (i.e., an F-shape), information can be  navigated more efficiently.
\citet{wang2014eye} show that the size of the fixation areas measured using eye-tracking is larger on web pages with more content, which increases the likelihood that the attention of a user is distracted.
%Finally, \citet{lindgaard2006attention} show that users are able to construct a stable judgment of a web page's visual appeal within 50ms. \todo{Is this important at all? I would drop the last reference and maybe use one or two others.}
The above-mentioned studies show the importance of the visual appearance of a web page and its effect on how users perceive pages.
This means that visual information has to be taken into account when ranking web pages given a user's query.

\citet{fan2017learning} are the first to approach the above problem.
The authors use snapshots of web pages to extract visual features for LTR
and show that such visual features significantly improve retrieval performance.
\citet{fan2017learning} feed snapshots through a neural network that attempts to model the previously mentioned F-shape.
The output of this neural network is then concatenated with more traditional content features, such as, e.g., BM25 and PageRank.
Finally, the proposed model (called ViP) is trained end-to-end by using a pairwise loss.
%\todo{Here, we have to briefly repeat what we have said in the introduction:
%there are two limitations in \cite{fan2017learning} (what limitations?), which we fix in this paper (how do we do that?).} 
The work of \citet{fan2017learning} is limited by: 
\begin{inparaenum}[(i)]
\item the GOV2 dataset, which lacks visual diversity, and 
\item not using state-of-the-art visual extraction methods.
\end{inparaenum}
We approach these problems by using a more diverse dataset based on the ClueWeb12 collection and improving the visual extraction methods by:
\begin{inparaenum}[(i)]
\item transfering pre-trained weights from a deep convolutional network by \citet{simonyan2014very}, and 
\item using synthetic saliency heatmaps from \citet{shan2017two} as input images.
\end{inparaenum}

%\todo{How is this relevance to our work apart from the fact that we use the method from \cite{shan2017two}?
%I think this paragraph can be dropped.}
%A number of techniques have been developed to predict saliency heatmaps on various images. \citet{buscher2009you} analyse the Web page's Document Object Model (DOM) to identify highly salient areas. More recent work from \citet{kummerer2016deepgaze} (on natural images) and \citet{shan2017two} (on web pages) use deep learning techniques to predict state-of-the-art saliency heatmaps. 


%In this work, we create a more generic approach by using synthetic generated saliency heatmaps.
%These heatmaps are used as an input to a convolution network in order to create features that can be used as an indicator of a web page communication effectiveness and usability. 

%\citet{donahue2014decaf} show that the features learned on large-scale supervised data can be transferred to different tasks and labels. Transferring the feature extraction weights to a new task is a common solution to cope with relatively small datasets. Most query sets used for LTR are relatively small (approximately 30,000 documents) compared to a dataset as ImageNet (1 million images), which indicates that transfer learning methods can be of use. In this work we use a pretrained image classifier and fine-tune its final layers on a LTR task. 

% Write something on how saliency can be used for evualuating web pages

% TODO: describe more related work.
% TODO: Show some work on visual features in web design
% TODO: 

% !TEX root = www2019-visual-ltr.tex

\section{\protect\modelname{} Model}
In this section, we introduce the \modelname{} model for \ac{LTR} with visual features.
The proposed model consists of three parts.
First, we introduce the model architecture in Section~\ref{sec:multimodal}.
Then, in Section~\ref{sec:visualfeatures} we describe two visual feature extractors used by \modelname{}:
VGG-16~\cite{simonyan2014very} and ResNet-152~\cite{he2016deep}, both pre-trained on ImageNet.
Finally, in Section~\ref{sec:saliency} we propose to enhance \modelname{} by generating synthetic saliency heatmaps for each of the input images.
%These synthetic saliency images are trained to learn a more expressive representation of the viewing pattern of a user. 
 The implementation of the proposed \modelname{} model is available online.\footnote{\label{coderef}\url{https://github.com/Braamling/learning-to-rank-webpages-based-on-visual-features}}

\subsection{Architecture} \label{sec:multimodal}
%In order to compare the performance of various visual feature extractor methods, we propose a reusable multimodal architecture. 
The \modelname{} architecture is visualized in Figure~\ref{fig:multimodelarchitecture}. 
The process starts by taking an image $x_i$ (1) as an input to the visual feature extraction layer (2) in order to create a generic visual feature vector $x_{vf}$. 
These features are considered generic because they can be extracted using convolutional filters trained on a different dataset and task. 
In order to transform generic visual features into \ac{LTR} specific visual features, we use $x_{vf}$ as an input to the visual feature transformation layer~(3).
This visual feature transformation layer outputs a visual feature vector $x_{vl}$ that can be used in combination with other \ac{LTR} features. 

Separating the visual feature extraction and transformation layers allows us to significantly reduce the computational requirements when using a pre-trained model for visual feature extraction. 
In Section~\ref{sec:sectionoptimalization} we further elaborate on how the computational requirements can be reduced.
In Section~\ref{sec:visualfeatures} below, we demonstrate how transfer learning can be applied to use pre-trained visual feature extraction methods in combination with the \modelname{} architecture.

The \modelname{} architecture also makes use of content features $x_{c}$~(4), e.g., BM25, PageRank, etc.
The final feature vector $x_{l}$ is constructed by concatenating the visual features $x_{vl}$ with the content features~$x_{c}$.
%
This final feature vector is then used as an input to the scoring component (5),
which transforms the features into a single score $x_s$ for each query-document pair.
The resulting model is trained end-to-end using a pairwise hinge loss with $L_2$ regularization similarly to~\cite{fan2017learning}.
The scoring component uses a single fully connected layer with a hidden size of $10$ and dropout set to $10\%$,
which showed good performance in preliminary experiments.

% \begin{multline}
% x_{vl} = f_{t}(f_{v}(x_{v})) \\ x_{s} = f_{s}(x_{vl} \oplus x_{c}) 
% \end{multline}


\begin{figure}[t]
\includegraphics[width = 3.4in]{images/multimodelarchitecture.pdf}
\caption{\modelname{} architecture.}
\label{fig:multimodelarchitecture}
\end{figure}

\subsection{Visual feature extractors} \label{sec:visualfeatures}
In order to use webpage snapshots in \ac{LTR}, the snapshots have to be converted to a vector representation that can then be used in combination with existing content features. 
Since preparing training data for \ac{LTR} is costly and its amount is usually low, it is beneficial to use a visual feature extraction method that is already pre-trained.
%on a large image dataset and for a different task, e.g., image classification.
%Convolutional filters generalize well and are easily transferable to other tasks.
In this work we use the VGG-16~\cite{simonyan2014very} and ResNet-152~\cite{he2016deep} models, two well-explored image classification models which both have an implementation with ImageNet pre-trained parameters.
Below, we describe how these two models are implemented within the \modelname{} architecture.

VGG-16~\cite{simonyan2014very} is commonly used for training transfer-learned models,
because it provides a reasonable trade-off between effectiveness and simplicity~\cite{shan2017two}.
Its architecture consists of a set of convolutional layers and fully connected layers. 
The convolutional layers extract features from an input image, which are then used by the fully connected layers to classify the image. 
The convolutional layers of VGG-16 are generic with respect to an input and task~\citep{donahue2014decaf}
and, thus, can be used as a visual feature extractor within the \modelname{} architecture to create generic visual features $x_{vf}$.
Hence, we use the convolutional layers as is, by freezing all the parameters during training.
Because we do not alter any of the convolutional layers, the size of $x_{vf}$ is determined by the output of the convolutional layers in the original VGG-16 model, being $1\times25088$.

The fully connected layers of VGG-16, instead, can be altered and retrained in order to be used with new inputs and tasks.
Due to this, we utilize them as a visual feature transformation layer within the \modelname{} architecture to produce \ac{LTR} specific features $x_{vl}$.
In particular, we replace the last fully connected layer of VGG-16 by a newly initialized fully connected layer.
Then we optimize the parameters of all fully connected layers of VGG-16 during training.
The size of $x_{vl}$ is set to $30$, as this size showed good performance in preliminary experiments.

The ResNet-152~\cite{he2016deep} architecture was shown to outperform VGG-16 in ImageNet classification.
The residual connections between convolutional layers of ResNet-152 allow for deeper networks to be trained without suffering from vanishing gradients.
Similarly to VGG-16, ResNet-152 has convolutional layers that extract features from an input image, which are in turn used by a fully connected layer to classify each image.
We use these convolutional layers as the visual feature extraction layer, which transforms $x_{i}$ to $x_{vf}$. All parameters of these convolutional layers are frozen during training. As with VGG-16, the size of $x_{vf}$ is determined by the output size of the original convolutional layers in the ResNet-152 model, $1\times2048$.


Additionally, the original ResNet-152 architecture only has a single fully connected layer, which empirically showed to not be enough to successfully train the \modelname~model.
Instead, we transform $x_{vf}$ to $x_{vl}$ by training a fully connected network from scratch.
The transformation layer is constructed using three layers with each having $4096$ hidden units and a final layer resulting in $x_{vl}$ with a size of $30$, which was empirically found to provide good performance in preliminary experiments.


\subsection{Saliency heatmaps} \label{sec:saliency}
In order to increase the ability to learn the visual quality of a webpage, we propose to explicitly model the user viewing pattern through synthetic saliency heatmaps.
The use of saliency heatmaps could be advantageous compared to the use of raw snapshots for the following reasons.
First, synthetic saliency heatmaps explicitly learn to predict how users perceive webpages by training an end-to-end model on actual eye-tracking data.
We expect this information to better correlate with webpage relevance compared to raw snapshots.
Second, saliency heatmaps reduce the average storage requirements by up to 90\%,
because they are gray-scale images and have large areas of the same color, which can be stored efficiently.
This makes the use of saliency heatmaps attractive for practical applications.
Figure~\ref{fig:exampleshots} shows example snapshots with their corresponding heatmaps (first and third columns respectively).

Following \cite{shan2017two}, we use a two-stage transfer learning model that learns how to predict saliency heatmaps on webpages.
Similarly to the visual feature extraction approaches above, \cite{shan2017two} takes a pre-trained image recognition model and finetunes the output layers on the following two datasets in order respectively:
\begin{inparaenum}[(i)]
\item SALICON~\cite{jiang2015salicon}, a large dataset containing saliency heatmaps created with eye-tracking hardware on natural images, and 
\item the webpage saliency dataset from \cite{shen2014webpage}, a smaller dataset containing saliency heatmaps created with eye-tracking hardware on webpages.
\end{inparaenum}

The trained model is used to convert a raw snapshot into a synthetic saliency heatmap. This heatmap is then used as an input image $x_i$ for the \modelname{} model (see Figure~\ref{fig:multimodelarchitecture}).
%
%The implementation of the \modelname{} model is available online.\footnote{\label{coderef}\url{https://github.com/Braamling/learning-to-rank-webpages-based-on-visual-features}}

\if0
The trained model is applied to the $3\times224\times224$ input images \todo{where do such images come from?}, resulting in grayscale heatmaps with a dimension of $1\times64\times64$.
\todo{Does this model always reduce the dimension size by 3?}
These heatmaps are then used as the an input image $x_{v}$ for the visual feature extractors described above (see Figure~\ref{fig:multimodelarchitecture}) by linearly scaling them to $3\times224\times224$, matching them with the VGG-16 and ResNet-152 input dimensions.
\fi
% !TEX root = cikm2018-visual-ltr.tex

\section{Visual ranking methods}
In this section we discuss the improved visual feature extraction methods that we use with \datasetname.

\subsection{Visual feature models}
Similarly to \cite{fan2017learning}, we use a combination of visual and content features for training.
This is achieved by separating the base model into a visual feature extraction and scoring component.
The visual feature extraction component takes an image $x_{img}$ as an input and outputs a visual feature vector $x_{v}$.
This feature vector is concatenated with a content feature vector $x_{c}$ and the concatenated vector is then used as an input to the scoring component.
In our case, the scoring component is a single fully connected layer with a hidden size of $10$ with dropout set to $10\%$.
\todo{Why these values? Same as in \cite{fan2017learning}?}
The combined models \todo{what do you mean by ``combined models'' here?} are trained end-to-end using pairwise hinge loss with $L_2$ regularization as in~\cite{fan2017learning}.
For the experiments, we used various visual feature extraction models described below.

\paragraph{ViP visual features}
As a baseline, we implement the visual feature extractor proposed in~\citet{fan2017learning} using PyTorch.
In this model, an input image $x_{img}$ is gray-scaled, normalized and horizontally segmented into $4\times16$ slices, which are processed separately through a shallow convolutional network.
This output is then passed top to bottom through an LSTM which results in a visual feature vector~$x_{v}$. 
%The full dimensions can be found in Figure \ref{fig:ViPfeat}.

\paragraph{VGG-16 visual features}
Since \datasetname~has a relatively low amount of snapshots to train a separate feature extractor, we use a pretrained ImageNet VGG-16 model~\cite{simonyan2014very} as a feature extractor.
\todo{Maybe say something about this being the state-of-the-art for visual feature extraction.}
These convolutional filters \todo{Which ``these'' filters? Also, we should either introduce what a ``convolutional filter'' is or replace this phrase} are generic with respect to an input and task, so we can reuse them to create visual features for LTR.
\todo{The next sentence is unclear to me and seems to be lacking context.}
During training, we only optimize the fully connected layers, of which the last one has been replaced by a reinitialized layer with an output of size $30$.
VGG-16 uses a $224\times224$ image with three color channels as an input
as opposed to \todo{XXX} used by ViP,
\todo{which gives us the following advantages: YYY.}

\paragraph{Saliency heatmaps}
Following \cite{shan2017two}, we generate synthetic saliency heatmaps for each snapshot in the \datasetname~data\-set.
The resulting heatmaps have a dimension of $64\times64$.
They are then linearly scaled to $224\times244$ and used as an input to the VGG-16 feature extraction component discussed above.
Figure \ref{fig:exampleshots} shows example saliency heatmaps with their corresponding snapshots.
\todo{Same here: why did we decide to do this? What does it give us?}

% !TEX root = www2019-visual-ltr.tex

\section{Experimental Setup}\label{sec:setup}
In this section, we discuss the configurations of the \modelname~architecture, baselines and metrics used during our experiments.

%The experiments are divided into two types of setups:
%\begin{inparaenum}[(i)]
%\item baseline experiments using only content features, and
%\item visual experiments using both content and visual features.
%\end{inparaenum}

\paragraph{\modelname~configurations}
We experiment with four configurations of the \modelname~architecture.
\modelname~baseline refers to the \modelname~model with only content features.
This configuration is trained by feeding content features into the scoring component directly, without adding any visual features.
\modelname~snapshots and \modelname~highlights use visual features extracted from vanilla snapshots of webpages and from snapshots of webpages with highlighted query terms, respectively.
Finally, \modelname~saliency uses visual features extracted from synthetic saliency heatmaps.

For the \modelname~configurations with visual features, we experiment with both VGG-16 and ResNet-152 visual feature extraction methods.
The learning rates for VGG-16 and ResNet-152 are set to and $0.0001$ and $0.00005$, respectively. 
Each experimental run is generated using the Adam optimizer~\cite{kingma2014adam} with a batch size of $100$.
These parameters are chosen based on preliminary experiments.


\paragraph{Baselines}
We compare our proposed \modelname~model to the ViP model by~\citet{fan2017learning}, the only existing \ac{LTR} method that uses visual features.
We train ViP on both vanilla and highlighted snapshots with the resulting configurations being ViP snapshots and ViP highlights, respectively.

Following~\cite{fan2017learning}, we also compare the \modelname~model to a number of content-based ranking methods,
namely BM25 and state-of-the-art \ac{LTR} techniques, such as RankBoost, AdaRank, and LambdaMart.\footnote{The \ac{LTR} methods are taken from \url{https://sourceforge.net/p/lemur/wiki/RankLib}.}


\paragraph{Metrics}
To measure the retrieval performance, we use precision and ndcg at $\{1,10\}$ and MAP.
Statistical significance is determined using a two-tailed paired t-test (p-value $\leq 0.05$). 

%All PyTorch experiments were performed on a single GTX 1080 Ti GPU with 11gb of RAM. 
% Preprocessing was performed on a Thinkpad X250 with an Intel i5-5300U CPU and 16gb of ram. 

% !TEX root = cikm2018-visual-ltr.tex

\begin{table}[h]
\caption{Comparison between the different visual \ac{LTR} methods. $\dagger$ indicates a significant improvement on the ViP baseline and $\ddagger$ indicates an improvement to the ViP baseline and highlights.}
\label{tab:visresults}
\centering
\begin{tabular}{llllll}
\toprule
                      & p@1    & p@10  & ndcg@1  & ndcg@10 & MAP   \\ 
\midrule
ViP baseline          & 0.338  & 0.370 & 0.189   & 0.233   & 0.415 \\ 
\midrule
ViP snapshots         & 0.392$^\dagger$ & 0.398$^\dagger$ & 0.217   & 0.254$^\dagger$   & 0.421 \\ 
ViP highlights        & 0.418$^\dagger$  & 0.416$^\dagger$ & 0.239$^\dagger$   & 0.269$^\dagger$   & 0.422 \\
\midrule
VGG snapshots      & 0.514$^\ddagger$    & 0.484$^\ddagger$ & 0.292$^\ddagger$   & 0.324$^\ddagger$   & 0.442$^\ddagger$ \\ 
VGG highlights     & 0.560$^\ddagger$    & 0.520$^\ddagger$ & 0.323$^\ddagger$   & 0.346$^\ddagger$   & 0.456$^\ddagger$ \\ 
\midrule
VGG saliency       & 0.554$^\ddagger$    & 0.453$^\ddagger$ & 0.310$^\ddagger$   & 0.302$^\ddagger$   & 0.422 \\ 
\bottomrule
\end{tabular}
\end{table}

\begin{table}[h]
\caption{Comparison of the VGG-16 model using highlights compared to several baselines. $*$ indicates a significant difference compared to the VGG-16 model. }

\label{tab:baseresults}
\begin{tabular}{llllll}
\toprule
                      & p@1    & p@10  & ndcg@1  & ndcg@10 & MAP   \\
\midrule
BM25                  & 0.300$^*$  & 0.316$^*$ & 0.153$^*$   & 0.188$^*$   & 0.350$^*$ \\ 
\midrule
RankBoost             & 0.450  & 0.444 & 0.258   & 0.288$^*$    & 0.427 \\
AdaRank               & 0.290$^*$   & 0.357$^*$  & 0.149$^*$    & 0.227$^*$    & 0.398 \\
LambdaMart            & 0.470  & 0.420$^*$ & 0.256   & 0.275$^*$    & 0.418 \\ 
\midrule
VGG highlights        & 0.560  & 0.520 & 0.323   & 0.346   & 0.456 \\ 
\bottomrule
\end{tabular}
\end{table}


\section{Results}
In this section, we show the results of the benchmark experiments and visual extraction methods discussed in section \ref{sec:experiments}. Significance has been determined using a two tailed paired t-test (p-value $\leq 0.05$). 

Table \ref{tab:11vs46} shows the results of the benchmark experiments. Each model was trained by optimizing the $P@5$ on the validation set. The results show that using the features from \datasetname~compared to the feature from LETOR has a minor impact on \ac{LTR} performance.  

Table \ref{tab:visresults} shows the results using the different visual ranking methods and the ViP baseline. 
Results marked with a $\dagger$ have a significance improvement compared to the ViP baseline model. 
We see that in terms of significance the results are similar to the work of \citet{fan2017learning}, where the screenshots with highlights yield better performance than using the vanilla screenshots. The $\ddagger$ indicates results where the VGG-16 model significantly outperforms the ViP model with highlighted screenshots. The results clearly show that introducing more powerful visual feature extraction methods significantly increases the \ac{LTR} performance.

Table \ref{tab:baseresults} compares the results of the VGG-16 model using highlighted screenshots with BM25, RankBoost, AdaRank and Lambdamart. The baselines with a significant difference from the VGG-16 model have been marked with a $*$. Although VGG-16 produces higher average results than the baseline methods, it is not possible to proof the statistical significance for some RankBoost and LambdaMart results. These results suggest that combining the visual features using more sophisticated ranking method could further improve \ac{LTR} performance using \datasetname. 
% !TEX root = cikm2018-visual-ltr.tex

\section{Conclusion}
% TODO write a piece about how these features can be used in a operational enviroment with having a big increase on performance requirements by storing $x_{ltr}$, r


In this paper, we introduced two feature extraction methods that significantly improve \ac{LTR} performance using visual features, and \datasetname, an out-of-the-box dataset for research on both visual and non-visual web page \ac{LTR}.
The experiments show that using state-of-the-art visual extraction methods can have a significant performance improvement compared to using only non-visual features. Although using VGG-16 as a feature extractor results in higher average results, there is no statistically significant improvement on some RankBoost and LambdaMart results. 

\if0
\todo{should we keep the following lines about masks in?} During this study we also explored using the highlights separate from the screenshots. However, this did not produce results worth mentioning. The dataset with separate highlights is available upon request. 
\fi

In future work it could be interesting to look at more ways to combine multiple visual features. Combining the features extracted from snapshots, highlights and saliency heatmaps could prove to improve ranking performance. 
A fairly straightforward future improvement would be to usage of more state-of-the-art ranking methods such as RankBoost and LambdaMart, which could improve ranking performance without changing the visual extraction method.

The recently introduced CapsuleNet \cite{sabour2017dynamic}, which is able to learn spatial relations in images, could potentially provide a significant increase \ac{LTR} performance when used as a visual feature extractor.  

Performing a more qualitative analysis on which elements cause the improvements in the visual \ac{LTR} methods would be interesting. The results would not only be useful for improving visual \ac{LTR}, but might also reveal new data driven design principles that can be used by content creators. 
% !TEX root = www2019-visual-ltr.tex

\section{Conclusion}
In this paper, we introduced two feature extraction methods that significantly improve \ac{LTR} performance using visual features, and the \datasetname{} dataset for research on both visual and non-visual webpage \ac{LTR}.
With the \datasetname{} it is now possible to easily develop and compare \ac{LTR} models with visual features. 

% The \datasetname{} dataset is the first publicly available dataset for \ac{LTR} with visual features and contains screenshots from both diverse and rich webpages. 
The experiments show that using state-of-the-art visual extraction methods can have a significant performance improvement compared to using only non-visual features. Additionally, we reach similar \ac{LTR} performance introducing synthetic saliency heatmaps which are $93.5\%$ smaller than the webpage snapshots. Finally, we show that the computational cost of adding the query-independent vanilla snapshots and saliency images to existing \ac{LTR} models is merely negligible.




% Although using VGG-16 as a feature extractor results in higher average results, there is no statistically significant improvement on some RankBoost and LambdaMart results. %TODO write something about the optimizations.

In future work, it could be interesting to look at more ways to combine multiple visual features. Combining the features extracted from snapshots, highlights and saliency heatmaps could further improve ranking performance. 
A fairly straightforward future improvement would be the usage of more state-of-the-art ranking methods such as RankBoost and LambdaMart, which could improve ranking performance without changing the visual extraction method.

The recently introduced CapsuleNet \cite{sabour2017dynamic}, which is able to learn spatial relations in images, could potentially provide a significant increase in \ac{LTR} performance when used as a visual feature extractor.  

\if0
During this study, we also explored using the highlights separate from the screenshots. 
However, this did not produce results worth mentioning. 
The dataset with separate highlights is available upon request. 
\fi

Performing a more qualitative analysis on which elements cause the improvements in the visual \ac{LTR} methods would be interesting. 
The results would not only be useful for improving visual \ac{LTR}, but might also reveal new data-driven design principles that can be used by content creators. 

\subsection*{Code and data}

Both the \datasetname~dataset\footnote{\url{https://github.com/Braamling/learning-to-rank-webpages-based-on-visual-features/blob/master/dataset.md}} and the code used to run the experiments in this paper\footnote{\url{https://github.com/Braamling/learning-to-rank-webpages-based-on-visual-features}} are available online.

\subsection*{Acknowledgements}
The Spark experiments in this work were carried out on the Dutch national e-infrastructure with the support of SURF Cooperative. Thanks to the Amsterdam Robotics Lab for using their computational resources. Thanks to Jamie Callan and his team for providing access to the online services for ClueWeb12. 

%\newpage

\bibliographystyle{ACM-Reference-Format}
\bibliography{cikm2018-visual-ltr} 

\end{document}
