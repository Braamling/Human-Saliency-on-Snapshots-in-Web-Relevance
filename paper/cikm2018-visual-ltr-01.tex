% !TEX root = cikm2018-visual-ltr.tex

\section{Introduction}
Over the years, various research demonstrated that web design is a determining factor for a user to divert to a competitors website~\cite{nielsen1999designing,nielsen2006f,pernice2017f,wang2014eye}.
%Over the years, various research demonstrated the impact of web design on how users find content on the web~\cite{nielsen1999designing,nielsen2006f,pernice2017f,wang2014eye}. 
Although web design has become a 34 billion dollar industry in the US alone according to \citet{ibisdesign}, modern search engines are still focused on using content features (e.g., BM25, PageRank, etc.) to determine web relevance \todo{refs?}. 

Recently \citet{fan2017learning} introduced ViP, a \ac{LTR} model that uses a combination of both visual and content features.
Visual features are calculated from snapshots, which are in turn created by rendering web pages.
The authors demonstrate that adding these visual features significantly improve the \ac{LTR} performance.
The indication that the visual representation of a web page can have a significant impact on web search leverages a new field in \ac{LTR} research.

However, there are a few limitations in \cite{fan2017learning}.
First, the rendered web pages come from the the GOV2 collection~\todo{ref}.\footnote{\url{http://ir.dcs.gla.ac.uk/test_collections/gov2-summary.htm}}
This collection is limited in the sense that:
\begin{inparaenum}[(i)]
\item it solely contains web pages within the .gov domain crawled in 2004, i.e., somewhat outdated pages with a relatively narrow scope,
\item original images are not included in the pages contained in the dataset, and
\item the styling information is not available together with web pages.
\end{inparaenum}
We believe that in order to advance research on \ac{LTR} with visual features it is required to have a dataset with more diverse and up-to-date documents and richer visual information (e.g., styling, images, etc).

A second limitation is the method used to extract visual features in the ViP architecture, which: 
\begin{inparaenum}[(i)]
\item is based on a strong assumption that users view websites in an F-shape pattern, 
\item grayscales and shrinks the input images to $64\times64$, and
\item does not use a state-of-the-art visual feature extraction method.
\end{inparaenum}

% \todo{a suboptimal / not state-of-the-art visual features extraction. Describe this in detail similarly to the above.}

In this work, we attempt to address the above limitations.
First, we propose \datasetname, a dataset that contains rich and highly diverse screenshots from the ClueWeb12 collection. The screenshots are taken from the judged documents in the TREC Web 2013 \& 2014 query sets. For each of the documents, we calculated content features that can be used for \ac{LTR}.
%\todo{describe key features of the proposed dataset by unfolding the next sentence into a few sentences}

We believe that this dataset allows the \ac{LTR} research community to investigate various methods of using visual features for search and ranking.
In particular, using \datasetname, we show that the results of the ViP model from \citet{fan2017learning} can be reproduced on a more diverse dataset.

Second, we use the following two state-of-the-art methods to extract visual features from snapshots:
\begin{inparaenum}[(i)]
\item transfer learning from a pre-trained image recognition model~\cite{donahue2014decaf,simonyan2014very}, and
\item generation of synthetic saliency heatmaps from the web page snapshots~\cite{shen2014webpage,shan2017two}.
\end{inparaenum}
We show that both methods are able to improve retrieval performance significantly.

The main contribution of this work are:
\begin{enumerate}  
\item We propose and publish \datasetname, an out-of-the-box dataset for \ac{LTR} with visual features.
\item We reproduce the ViP model from \cite{fan2017learning} on the newly proposed dataset.
\item We improve ViP by applying a transfer learning feature extractor and using synthetic saliency image as an input.
\end{enumerate}

\todo{Revise this if we have space or drop it.} The rest of the paper is organized as follows. Section~\ref{sec:dataset} describes the collection process that was used to construct \datasetname. In Section~\ref{sec:experiments} we reproduce the work of \citet{fan2017learning} and demonstrate various feature extraction methods on \datasetname~and set a baseline for future visual \ac{LTR} research.  