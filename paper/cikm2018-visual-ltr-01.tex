% !TEX root = cikm2018-visual-ltr.tex

\section{Introduction}
Over the years, various research demonstrated the impact of web design on how users find content on the web~\cite{nielsen1999designing,nielsen2006f,pernice2017f,wang2014eye}.
Although web design has become a 34 billion dollar industry in the US alone according to \citet{ibisdesign}, modern search engines are still focused on using content features (e.g., BM25, PageRank, etc.) to determine web relevance \todo{refs?}. 

Recently \citet{fan2017learning} introduced ViP, a \ac{LTR} model that uses a combination of both visual and content features.
Visual features are calculated from snapshots, which are in turn created by rendering web pages.
The authors demonstrate that adding these visual features significantly improve the \ac{LTR} performance.
The indication that the visual representation of a web page can have a significant impact on web search leverages a new field in \ac{LTR} research.

However, there are a few limitations in \cite{fan2017learning}.
First, the rendered web pages come from the the GOV2 collection~\todo{ref}.\footnote{\url{http://ir.dcs.gla.ac.uk/test_collections/gov2-summary.htm}}
This collection is limited in the sense that:
\begin{inparaenum}[(i)]
\item it solely contains web pages within the .gov domain crawled in 2002, i.e., somewhat outdated pages with a relatively narrow scope,
\item original images are not included in the pages contained in the dataset, and
\item the styling information is not available together with web pages.
\end{inparaenum}
Considering these limitations, further visual \ac{LTR} research needs a dataset created from a more diverse document collections, which can be used out-of-the-box together with prepared judgments, content features and screenshots. 

Second limitation is \todo{a suboptimal / not state-of-the-art image processing method}

In this work, we propose \datasetname, a dataset that allows the \ac{LTR} research community to investigate various methods of using visual features in web relevance. \datasetname is a combination of queries and content features from TREC Web 2013 \& 2014 together with screenshots from ClueWeb12 documents. 

Using \datasetname we showed that the results from \citet{fan2017learning} can be reproduced on a more diverse dataset. Moreover, we demonstrate the following two methods that can be used to extract visual features from web screenshots: i) using transfer learning from a pre-trained image recognition model~\cite{donahue2014decaf}\cite{simonyan2014very} and ii) by creating a synthetic saliency heatmap from the web page screenshot~\cite{shen2014webpage}\cite{shan2017two}, which were both able to make a significant improvement in retrieval performance. 

The main contribution of this work are:
\begin{enumerate}  
\item An out-of-the-box dataset for visual \ac{LTR} research.
\item A demonstration of the abilities of visual features in \ac{LTR}.
\item A baseline for future visual \ac{LTR} research.
\end{enumerate}

Section \ref{sec:dataset} describes the collection process that was used to construct \datasetname. In section \ref{sec:experiments} we reproduce the work of \citet{fan2017learning}, demonstrate various feature extraction methods on \datasetname and set a baseline for future visual \ac{LTR} research.  