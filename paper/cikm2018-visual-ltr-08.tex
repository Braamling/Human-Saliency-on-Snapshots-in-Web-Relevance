% !TEX root = www2019-visual-ltr.tex

\section{Conclusion}
In this paper, we considered the problem of \ac{LTR} with visual features.
We proposed the \modelname~model that extracts visual features from webpage snapshots
using transfer learning (specifically, pre-trained VGG-16 and ResNet-152 models) and synthetic saliency heatmaps.
In both cases the extracted visual features significantly improved the \ac{LTR} performance.
We also showed that the proposed \modelname~model significantly outperformed the visual \ac{LTR} baselines.

In addition to the model, we released the \datasetname~dataset, containing visually rich and diverse webpages with corresponding snapshots.
With the \datasetname~dataset it is now possible to comprehensively study \ac{LTR} with visual features.

One direction for future work is the study of state-of-the-art \ac{LTR} methods, such as RankBoost and LambdaMart, within the scoring components of the \modelname~model.
This could improve performance without changing the visual feature extraction and transformation components.
Another promising direction for future work is to combine multiple visual features,
i.e., visual features extracted from vanilla snapshots, snapshots with highlights and saliency heatmaps. Other visual feature extractors, such as the CapsuleNet~\cite{sabour2017dynamic} model, which is able to learn spatial relations in images, might also provide additional performance. Other methods of combining visual and textual features, such as text-image co-embeddings, might also be worth exploring. 
Future work could also investigate the robustness of this method when using various rendering variations, e.g., different browser, resolutions. 
Finally, performing a more qualitative analysis on which elements cause improvements in \ac{LTR} with visual features would be interesting, e.g., by interpreting the filters in the feature extractor~\cite{olah2018the}. 
%The results 
%would not only be useful for improving the \ac{LTR} performance, but 
%might also reveal new data-driven design principles that can be used by content creators. 

\if0
During this study, we also explored using the highlights separate from the screenshots. 
However, this did not produce results worth mentioning. 
The dataset with separate highlights is available upon request. 
\fi