% !TEX root = cikm2018-visual-ltr.tex

\section{Conclusion}
In this paper, we introduced two feature extraction methods that significantly improve \ac{LTR} performance using visual features, and the \datasetname{} dataset for research on both visual and non-visual webpage \ac{LTR}.
The experiments show that using state-of-the-art visual extraction methods can have a significant performance improvement compared to using only non-visual features. Additionally, we reach similar \ac{LTR} performance introducing synthetic saliency heatmaps which are $93.5\%$ smaller than the webpage snapshots. Finally, we show that the computational cost of adding the query-independent vanilla snapshots and saliency images to existing \ac{LTR} models is merely negligible.

% Although using VGG-16 as a feature extractor results in higher average results, there is no statistically significant improvement on some RankBoost and LambdaMart results. %TODO write something about the optimizations.

In future work, it could be interesting to look at more ways to combine multiple visual features. Combining the features extracted from snapshots, highlights and saliency heatmaps could further improve ranking performance. 
A fairly straightforward future improvement would be the usage of more state-of-the-art ranking methods such as RankBoost and LambdaMart, which could improve ranking performance without changing the visual extraction method.

The recently introduced CapsuleNet \cite{sabour2017dynamic}, which is able to learn spatial relations in images, could potentially provide a significant increase in \ac{LTR} performance when used as a visual feature extractor.  
During this study, we also explored using the highlights separate from the screenshots. 
However, this did not produce results worth mentioning. 
The dataset with separate highlights is available upon request. 
Performing a more qualitative analysis on which elements cause the improvements in the visual \ac{LTR} methods would be interesting. 
The results would not only be useful for improving visual \ac{LTR}, but might also reveal new data-driven design principles that can be used by content creators. 